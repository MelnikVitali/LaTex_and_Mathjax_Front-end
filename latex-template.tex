\documentclass[12pt]{article}   % you have 10pt, 11pt, or 12pt options

\setlength{\textwidth}{17.2cm}     % if you change this, consider changing
\setlength{\evensidemargin}{-.3cm} % side margins to retain centering
\setlength{\oddsidemargin}{-.3cm}

\setlength{\textheight}{24cm}   % if you change this, consider changing
\setlength{\topmargin}{-2.5cm}  % top margin to retain centering
\setlength{\headsep}{1.6cm}

\usepackage{amsmath, latexsym, graphicx, enumerate}

% This section defines all the environments you might use.  Just type
% \begin{theorem, or corollary, or whatever}, then the optional name of the
% theorem inside {} (or empty {} if no name), then body of the theorem,
% corollary, whatever, also inside {} then \end{theorem, corollary, whatever}

\newenvironment{theorem}[1]{\vspace{.9cm}\noindent
    {\bf Theorem {#1}}}{\vspace{.1cm}}
\newenvironment{lemma}[1]{\vspace{.9cm}\noindent
    {\bf Lemma {#1}}}{\vspace{.1cm}}
\newenvironment{corollary}[1]{\vspace{.9cm}\noindent
    {\bf Corollary {#1}}}{\vspace{.1cm}}
\newenvironment{definition}{\vspace{.9cm}\noindent
    {\bf Definition}}{\vspace{.1cm}}
\def\qed{\hfill $\Box$}
\newenvironment{proof}{\vspace{.5cm}
    \noindent{\bf Proof: }}{\qed \vspace{1cm}}

\DeclareSymbolFont{AMSb}{U}{msb}{m}{n}  % the symbols below will give you the
                    % blackboard bold of R, T, etc.
\DeclareMathSymbol{\Sph}{\mathbin}{AMSb}{"53} \DeclareMathSymbol{\R}{\mathbin}{AMSb}{"52}
\DeclareMathSymbol{\T}{\mathbin}{AMSb}{"54} \DeclareMathSymbol{\Z}{\mathbin}{AMSb}{"5A}
\DeclareMathSymbol{\K}{\mathbin}{AMSb}{"4B}\DeclareMathSymbol{\C}{\mathbin}{AMSb}{"43}



\begin{document}            % necessary part of document

\centerline{\Large{\LaTeX \, Tutorial for Math Majors}} % centers the text, writes in Large font

\vspace{1cm}

Here are some typesetting features you may want to use when writing up your classwork, or
the mathematics in your class summary. If you want to type a paragraph of text, simply
start typing.

To start a new paragraph, leave a blank line before the new paragraph.

\bigskip

Here's a bullet list of some of the math symbols you may need.  Note that any math
formulas must be surrounded by dollar signs, like so:  $H(s,t) = F(\alpha(s),t)$.  If you
surround a math formula by double dollar signs, your formula will be centered on a line by
itself, like so:
$$H(s,t) = F(\alpha(s),t).$$  Whatever you type afterwards will begin again on a separate line.

\begin{itemize}     % replace itemize with "enumerate" for numbered list
\item Greek letters:  $\alpha, \gamma, \pi, \tau$
\item product of two sets $X \times Y$
\item Intersections $\cap$, unions $\cup$, and disjoint unions $\sqcup$
\item {\it italics} and {\bf bold}
\item related to: $\sim$, homotopic to $\simeq$, and isomorphic to $\cong$
\item Fractions which fit inside a paragraph of text:
$\frac{az + b}{cz + d}$,
and bigger fractions: $\displaystyle{\frac{az + b}{cz + d}}$
\item Subscripts and exponents:  $z_1$, $w^2$, $z_2^3$, $f_*(x)$,  $p^{-1}(b)$
\item Derivatives: $f'(x)$, integrals $\int_a^b f(x) \, dx$, % the \, adds a space between f(x) and dx
and limits $\lim_{n \rightarrow \infty} a_n$ or $\displaystyle{\lim_{n \rightarrow \infty} a_n}$
\item Not equals:  $c \neq 0$, or greater than / less than or equal: $c \ge 0$, $x \le 17$
\item functions defined in pieces:  $$p(x) =  \left\{ \begin{array}{ll}
                x & \mbox{for } x \in [0,1] \\
                x-1 & \mbox{for } x \in [2,3]
                \end{array} \right.$$
\item Left quotes `` and right quotes "
\item Composition:  $g \circ f$, and multiplication:  $g \cdot f$
\item Left and Right Set Brackets need a backslash:  $\{x : p(x) = b\}$
\item Is an element of: $b \in B$
\item $\R$,  $\Sph^2$, $\T^2$, $\Z$
\item To put a word in with a string of math symbols, use mbox:  $f \sim g \mbox{ rel } A$,
otherwise, it looks like: $f \sim g rel A$.
\item $p|_{\widetilde{U}}$
\item group presentation:  $\langle a,b : ab\overline{a} \rangle$
\item A lot of symbols you might want to know are just what you think
they might be, preceded by a backslash:  $\cos \theta, \not\in, \rightarrow, \mapsto,
\Leftrightarrow, \longrightarrow, \subset, \subseteq$
\end{itemize}


\noindent There are nice, pre-written environments for Theorems and Proofs,
as below:

\begin{theorem}{(Unique Path Lifting Property)}
% leave this as empty brackets : {} if theorem has no name
Here's where you type in the text of the theorem.
\end{theorem}

\begin{proof}
And this is where you type in the proof!
\end{proof}

\begin{lemma}
{} % this lemma has no name
Here's where you put the body of a lemma.
\end{lemma}

\bigskip

\noindent You might also want to write up the following things:

\begin{enumerate}
\item A numbered list,

\item or a sequence of equations, lined up at the equals sign, with comments...
\begin{align*}   % do not need dollar signs because eqnarray puts you in math mode
d(z_1,z_2) & =  \int_{z_1}^{z_2} \frac{1}{t} \, dt \\
& =  \ln(z_2) - \ln(z_1) & \mbox{by the Fund Thm of Calc}\\
& =  \ln\left(\frac{z_1}{z_2}\right) \\ % using \left( and \right) makes larger ()
\end{align*}

\item or numbered equations, so you can refer to them later, like equation \ref{this-one}
\begin{align}
\frac{dI}{dt} &=  \alpha S I - \beta I \\
&=  \alpha I \left(S - \frac{\beta}{\alpha}\right) \label{this-one}\\
&=  \alpha I \left(N  - I  - \frac{\beta}{\alpha}\right)  \\
\end{align}

\item or some Commutative Diagrams...

$$\begin{array}{rcl}
\Sph^2  & \stackrel{g}{\longrightarrow} & \Sph^2 \\
\mbox{\footnotesize{S}} \downarrow & & \downarrow \mbox{\footnotesize{S}} \\
\R^2 & \stackrel{f}{\longrightarrow} & \R^2 \\
\end{array}$$

\item or a Table...

\begin{center}
\begin{tabular}{|c|c|}
\hline 
Column A & Column B \\ 
\hline
$T^2 \# S^2$ & $P^2 \# P^2$ \\
$K^2$ & $K^2 \# P^2$ \\
$S^2 \# S^2 \# S^2$ & $S^2 \# S^2$ \\
$P^2 \# T^2$ & $P^2 \# P^2 \# P^2 \# K^2$ \\
$K^2 \# T^2 \# P^2$ & $T^2$ \\
\hline
\end{tabular}
\end{center}

\item or a picture, such as in Figure \ref{fig:fun} (you will need to use a .eps graphics file for Windows, and a .pdf graphics
file for Mac).

\end{enumerate}

Maybe you want a big vertical skip

\vspace{1cm}

\noindent Or maybe you want a more specialized list....like letters in parentheses, or roman numbers with a slash: the following two require adding the word \texttt{enumerate} as a package  in the header (see the top of this file)

\begin{enumerate}[(a)]
\item first thing
\item second thing
\end{enumerate}

or

\begin{enumerate}[I/]
\item first thing
\item second thing
\end{enumerate}

\noindent ...and you can always ask me if you need to typeset something that I haven't
included here.


\end{document}              % necessary part of the document
